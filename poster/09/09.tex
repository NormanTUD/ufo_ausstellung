\documentclass{scrartcl}

% These are only some keywords for the autocompletion-feature of many editors: section, subsection, subsubsection, paragraph,
% includegraphics, width, linewidth, linespread, figure, wrapfigure, caption, label, footnote, equation, input, cite, citetitle,
% citeauthor, footfullcite, tableofcontents, printbibliography, clearpage, frq, frqq, flq, flqq, grq, grqq, glq, glqq, textit,
% texttt, mathrm, dots, pmatrix, centering, phantom, minipage, ensuremath, hfill, vfill, 

\newcommand{\centeredquote}[2]{
	\hbadness=5000
	\vspace{-1em}
	\begin{flushright}
		\item\frqq\textsl{#1}\flqq\ 
	\end{flushright}
	\nopagebreak
	\hfill ---\,\textsc{#2}\newline
	\vspace{-1em}
}

\newcommand{\centeredquoteunknownsource}[1]{
	\hbadness=5000
	\vspace{-1em}
	\begin{quotation}
		\begin{flushright}
			\item\frqq\textsl{#1}\flqq\ 
		\end{flushright}
	\end{quotation}
	\vspace{-1em}
}
\usepackage[utf8]{inputenc}
\usepackage[T1]{fontenc}
\usepackage[sc,osf]{mathpazo}
\usepackage{fourier}
\usepackage{soulutf8}
\usepackage{graphicx}
\usepackage{amsmath}
\usepackage{amssymb}
\usepackage[ngerman]{babel}
\usepackage{tikz}
\usepackage{pgfplotstable}
\usepackage{ifthen}

\usepackage{qrcode}

\emergencystretch2em

\newcommand{\qr}[2][]{%
    \noindent % Verhindert Einrückung der Zeile
    \mbox{\qrcode[height=0.6in]{#2}\ifthenelse{\equal{#1}{}}{}{\textit{(#1)\quad}}} % Optionalen Text kursiv unter dem QR-Code
}

\usepackage{geometry}
\newgeometry{left=0.1cm,bottom=0.1cm,top=0.1cm,right=0.1cm}


\begin{document}

\textit{Von oben nach unten und links nach rechts}:

\begin{itemize}
	\item Die erste Reihe zeigen Bilder eines UFOs, das angeblich in Nashville gesichtet worden sein soll. In Wahrheit ist es ein Stück Bühnendekoration für ein \textit{Electronic-Light-Orchestra}-Konzert (das zeigt das Bild darunter).

		\qr[UFO-Case-Book-Artikel über die 1989 Nashville UFO-Bilder]{https://www.ufocasebook.com/1989nashvillephotographs.html}
		\qr[Artikel über die Konzerte von ELO und deren Props]{http://www.jefflynnesongs.com/ootbtour/}
	\item Rechts daneben ein Bild eines UFOs, von denen 1967 in England auf vielen Feldern welche aufgetaucht sind. Sie waren gefüllt mit schwarzer Flüssigkeit. Es stellte sich heraus: es war ein studentischer Scherz.

		\qr[Wikipedia-Eintrag zum 1967 British Flying Saucer Hoax]{https://en.wikipedia.org/wiki/1967_British_flying_saucer_hoax}
	\item Darunter ein Bild einer Voyager-Test-Sonde in der White-Sands-Missile-Range in New Mexiko. Es wird oft als Bild zum Thema UFO benutzt, ohne dass erklärt wird, was es wirklich war. Das Bild wurde offiziell von der NASA veröffentlicht.

		\qr[Artikel über Roswell, bei der das Bild erwähnt wird]{https://www.af.mil/The-Roswell-Report/}
	\item Darunter das sogenannte Calvine-UFO. Angeblich wurde es von einem Jagdflugzeug verfolgt. Höchstwahrscheinlich ist es eine Fotomanipulation, und das UFO eigentlich die Spitze eines Berges.

		\qr[Artikel zum Calvine-UFO]{https://www.express.co.uk/news/weird/1947448/Calvine-ufo-breakthrough-solved}
\end{itemize}

\end{document}
