\documentclass{scrartcl}

% These are only some keywords for the autocompletion-feature of many editors: section, subsection, subsubsection, paragraph,
% includegraphics, width, linewidth, linespread, figure, wrapfigure, caption, label, footnote, equation, input, cite, citetitle,
% citeauthor, footfullcite, tableofcontents, printbibliography, clearpage, frq, frqq, flq, flqq, grq, grqq, glq, glqq, textit,
% texttt, mathrm, dots, pmatrix, centering, phantom, minipage, ensuremath, hfill, vfill, 

\newcommand{\centeredquote}[2]{
	\hbadness=5000
	\vspace{-1em}
	\begin{flushright}
		\item\frqq\textsl{#1}\flqq\ 
	\end{flushright}
	\nopagebreak
	\hfill ---\,\textsc{#2}\newline
	\vspace{-1em}
}

\newcommand{\centeredquoteunknownsource}[1]{
	\hbadness=5000
	\vspace{-1em}
	\begin{quotation}
		\begin{flushright}
			\item\frqq\textsl{#1}\flqq\ 
		\end{flushright}
	\end{quotation}
	\vspace{-1em}
}
\usepackage[utf8]{inputenc}
\usepackage[T1]{fontenc}
\usepackage[sc,osf]{mathpazo}
\usepackage{fourier}
\usepackage{soulutf8}
\usepackage{graphicx}
\usepackage{amsmath}
\usepackage{amssymb}
\usepackage[ngerman]{babel}

\usepackage{qrcode}

\emergencystretch2em

\newcommand{\qr}[1]{
	\qquad \qrcode[height=0.4in]{#1}
}

\usepackage{geometry}
\newgeometry{left=0.1cm,bottom=0.1cm,top=0.1cm,right=0.1cm}

\begin{document}

\textit{Von oben nach unten und links nach rechts}:

\begin{itemize}
	\item Oben ein Zeitungsartikel aus einer kanadischen Zeitung, die Lentikularwolken als UFOs fehldeutet und dramatisch aufarbeitet.
	\item Darunter zwei Bilder von Sternen, herangezoomt. Durch chromatische Aberration verschwimmen die Sterne und werden bunt und flackern. Sterne werden von UFO-Gläubigen oft photographiert und als Beweis für UFOs gesehen, denn sie sehen eigenartig aus. Das reicht für sie, um zu beweisen, dass die NASA, ESA usw. lügen, und Sterne keine Sonnen sind.
	\item Darunter ein Bild aus einem Nachrichtenschnipsel einer Nachrichtensendung aus Bagdad. Darauf zu sehen ist ein langes Wesen, sogenannte Flying Rods, die auch auf dem Bild daneben zu sehen sind. Flying Rods sind in Wirklichkeit Insekten, die schnell durchs Bild fliegen und daher verschwommen und langgezogen werden.
	\item Darunter ist ein Bild eines Lens Flares, eines Linseneffektes, die auch oft für UFOs gehalten werden.
	\item Darunter ein Bild von Sternschnuppen, die auch oft für UFO-Sichtungen sorgen.
\end{itemize}

\end{document}
