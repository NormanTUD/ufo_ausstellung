\documentclass{scrartcl}

% These are only some keywords for the autocompletion-feature of many editors: section, subsection, subsubsection, paragraph,
% includegraphics, width, linewidth, linespread, figure, wrapfigure, caption, label, footnote, equation, input, cite, citetitle,
% citeauthor, footfullcite, tableofcontents, printbibliography, clearpage, frq, frqq, flq, flqq, grq, grqq, glq, glqq, textit,
% texttt, mathrm, dots, pmatrix, centering, phantom, minipage, ensuremath, hfill, vfill, 

\newcommand{\centeredquote}[2]{
	\hbadness=5000
	\vspace{-1em}
	\begin{flushright}
		\item\frqq\textsl{#1}\flqq\ 
	\end{flushright}
	\nopagebreak
	\hfill ---\,\textsc{#2}\newline
	\vspace{-1em}
}

\newcommand{\centeredquoteunknownsource}[1]{
	\hbadness=5000
	\vspace{-1em}
	\begin{quotation}
		\begin{flushright}
			\item\frqq\textsl{#1}\flqq\ 
		\end{flushright}
	\end{quotation}
	\vspace{-1em}
}
\usepackage[utf8]{inputenc}
\usepackage[T1]{fontenc}
\usepackage[sc,osf]{mathpazo}
\usepackage{fourier}
\usepackage{soulutf8}
\usepackage{graphicx}
\usepackage{amsmath}
\usepackage{amssymb}
\usepackage[ngerman]{babel}
\usepackage{tikz}
\usepackage{pgfplotstable}
\usepackage{ifthen}

\usepackage{qrcode}

\emergencystretch2em

\newcommand{\qr}[2][]{%
    \noindent % Verhindert Einrückung der Zeile
    \mbox{\qrcode[height=0.6in]{#2}\ifthenelse{\equal{#1}{}}{}{\textit{(#1)\quad}}} % Optionalen Text kursiv unter dem QR-Code
}

\usepackage{geometry}
\newgeometry{left=0.1cm,bottom=0.1cm,top=0.1cm,right=0.1cm}


\begin{document}

\textit{Von oben nach unten und links nach rechts}:

\begin{itemize}
	\item Oben ein Zeitungsartikel aus einer kanadischen Zeitung, die Lentikularwolken als UFOs fehldeutet und dramatisch aufarbeitet. Eigentlich entstehen diese Wolkenformen durch Windströmungen, die feuchte Luft anheben. \qr[Wikipedia-Eintrag zu Lentikularwolken]{https://de.wikipedia.org/wiki/Lenticularis}
	\item Die beiden Bilder zeigen Sterne, die durch chromatische Aberration in ein buntes Spektakel verwandelt werden. Astronomen kennen dieses Phänomen: Wenn Licht durch ein Linsensystem bricht, können die verschiedenen Wellenlängen – also die Farben – unterschiedlich stark gebrochen werden, was zu einem verwischten, flackernden Bild führt. UFO-Gläubige hingegen interpretieren diese verwirrenden Lichter oft als Beweis für eine geheime Agenda der NASA oder ESA. Es ist amüsant, dass das, was einfach nur die Grenzen der Optik demonstriert, als Verschwörungstheorie verkauft wird. Man fragt sich, ob ein wirkliches UFO nicht vielleicht nur ein besonders verwirrter Stern ist! \qr[Video: Heranzoomen an einen Stern]{https://www.youtube.com/watch?v=ZOwcvv034Ho} \qr[Wikipedia-Eintrag über chromatische Aberration]{https://de.wikipedia.org/wiki/Chromatische_Aberration}
	\item Bild eines langen Wesens (\textit{Flying Rod}): In einem Nachrichtenschnipsel aus Bagdad sehen wir ein langes Wesen, das als Flying Rod bekannt ist. Diese vermeintlichen UFOs sind in Wahrheit nichts anderes als Insekten, die in rasanter Geschwindigkeit durch den Bildausschnitt fliegen und dabei durch die Kameraaufnahme verzerrt werden. Der schnelle Flug macht sie lang und schlank, und schon hat man das perfekte „UFO“ für die nächste Verschwörungstheorie. Man kann sich vorstellen, wie sehr sich die Kritiker amüsieren, wenn sie die Debatte um die „überlegene Technologie“ der Flying Rods führen – eine Technologie, die offensichtlich nicht viel mehr ist als ein schnelles Insekt und ein unglückliches Kameraobjektiv. \qr[Wikipedia-Eintrag zu Flying Rods]{https://en.wikipedia.org/wiki/Rod_(optical_phenomenon)} \qr[Skeptoid-Episode über Flying Rods]{https://skeptoid.com/episodes/4004}
	\item Lens Flares sind ein weiteres Phänomen, das oft für UFOs gehalten wird. Diese Linseneffekte entstehen, wenn Lichtstrahlen durch die Linse einer Kamera gebrochen werden und dabei interessante, oftmals kreisförmige Muster erzeugen. Oft sieht man in diesen Effekten alles Mögliche: von schimmernden \frq Raumschiffen\flq\ bis hin zu geheimnisvollen \flq Lichtern\frq. Es ist faszinierend, wie selbst ein kleiner Lichtreflex ausreicht, um die Vorstellungskraft vieler zu beflügeln und sie zu einer intergalaktischen Debatte zu inspirieren. Vielleicht sollten wir den Lens Flare als neuen, \frq beweisenden\flq\ Beweis für UFOs einführen – schließlich war er schon in so vielen Filmen ein Star! \qr[Wikipedia-Eintrag zu Lens-Flares]{https://en.wikipedia.org/wiki/Lens_flare}
	\item Darunter ein Bild einer Sternschnuppe, auch bekannt als Meteor, die schon seit Jahrhunderten die Menschen verzaubert haben und oft als Vorboten von Veränderungen interpretiert wurden. Die glühenden Lichtstreifen, die durch die Erdatmosphäre rasen, werden häufig fälschlicherweise als UFO-Sichtungen deklariert. Diese kleinen Stücke aus dem All, die in einem Hauch von Staub und Glanz verglühen, sind der perfekte Aufhänger für all jene, die das Außergewöhnliche suchen. Man fragt sich, wie oft eine gute Sternschnuppe für den nächsten großen UFO-Skandal verantwortlich ist. \qr[Wikipedia-Eintrag zu Meteoren]{https://de.wikipedia.org/wiki/Meteor}
\end{itemize}

\end{document}
