\documentclass{scrartcl}

% These are only some keywords for the autocompletion-feature of many editors: section, subsection, subsubsection, paragraph,
% includegraphics, width, linewidth, linespread, figure, wrapfigure, caption, label, footnote, equation, input, cite, citetitle,
% citeauthor, footfullcite, tableofcontents, printbibliography, clearpage, frq, frqq, flq, flqq, grq, grqq, glq, glqq, textit,
% texttt, mathrm, dots, pmatrix, centering, phantom, minipage, ensuremath, hfill, vfill, 

\newcommand{\centeredquote}[2]{
	\hbadness=5000
	\vspace{-1em}
	\begin{flushright}
		\item\frqq\textsl{#1}\flqq\ 
	\end{flushright}
	\nopagebreak
	\hfill ---\,\textsc{#2}\newline
	\vspace{-1em}
}

\newcommand{\centeredquoteunknownsource}[1]{
	\hbadness=5000
	\vspace{-1em}
	\begin{quotation}
		\begin{flushright}
			\item\frqq\textsl{#1}\flqq\ 
		\end{flushright}
	\end{quotation}
	\vspace{-1em}
}
\usepackage[utf8]{inputenc}
\usepackage[T1]{fontenc}
\usepackage[sc,osf]{mathpazo}
\usepackage{fourier}
\usepackage{soulutf8}
\usepackage{graphicx}
\usepackage{amsmath}
\usepackage{amssymb}
\usepackage[ngerman]{babel}
\usepackage{tikz}
\usepackage{pgfplotstable}

\usepackage{qrcode}

\emergencystretch2em

\newcommand{\qr}[1]{
	\qquad \qrcode[height=0.6in]{#1}
}

\usepackage{geometry}
\newgeometry{left=0.1cm,bottom=0.1cm,top=0.1cm,right=0.1cm}


\begin{document}

\textit{Von oben nach unten und links nach rechts}:

\begin{itemize}
	\item Ein angebliches Bild eines Aliens, das Hitler trifft. Das Originalbild zeigt Hitler bei einem Handschlag mit Mussolini.
	\item Die deutlichen \frq UFO\flq-Bilder sind die angeblich von den Nazi entwickelten Raumfahrzeuge \textit{Haunebu}, \textit{Haunebu II} und \textit{Vril}, von denen außerhalb einiger angeblicher Bilder aus dubiosen Quellen und \frq Baupläne\flq\ nichts historisch belegt ist. Ganz unten ist der angebliche Bauplan des \frq Andromeda-Geräts\flq, einer Art Mutterschiff für die Nazi-UFOs.
	\item Das Bild in der Mitte zeigt \textit{Dr. Axel Stoll}, der bekannt wurde als rechtsesoterischer Verschwörungstheoretiker. Er wurde einem breiterem Publikum bekannt durch die auf Youtube hochgeladenen Aufnahmen seines \textit{Neuschwabenlandtreffens}, einer Gruppe von Rechtsesoterikern, die sich regelmäßig in einer Kneipe getroffen haben und über ihre Thesen geredet haben.

		\centeredquote{Die weiße Rasse stammt vom Aldebaran, vergesst das nicht!}{Dr. Axel Stoll}

		\qr{https://archive.ph/X4UOz\#selection-976.0-976.1}
		\qr{https://de.wikipedia.org/wiki/Axel_Stoll}
	\item Das Bild mit dem weißen Kreis und dem Flugzeugflügel zeigt einen \textit{Foo Fighter}, eine Art \frq UFO\flq, wie sie von allen Seiten im zweiten Weltkrieg berichtet worden. Die Nazis dachten, es wäre eine Geheimwaffe der Amerikaner und umgekehrt. Diese Kugeln sollen über längere Strecken Flugzeuge verfolgt und sich \frq intelligent\flq\ verhalten haben. \qr{https://en.wikipedia.org/wiki/Foo_fighter}
\end{itemize}

\end{document}
