\documentclass{scrartcl}

% These are only some keywords for the autocompletion-feature of many editors: section, subsection, subsubsection, paragraph,
% includegraphics, width, linewidth, linespread, figure, wrapfigure, caption, label, footnote, equation, input, cite, citetitle,
% citeauthor, footfullcite, tableofcontents, printbibliography, clearpage, frq, frqq, flq, flqq, grq, grqq, glq, glqq, textit,
% texttt, mathrm, dots, pmatrix, centering, phantom, minipage, ensuremath, hfill, vfill, 

\newcommand{\centeredquote}[2]{
	\hbadness=5000
	\vspace{-1em}
	\begin{flushright}
		\item\frqq\textsl{#1}\flqq\ 
	\end{flushright}
	\nopagebreak
	\hfill ---\,\textsc{#2}\newline
	\vspace{-1em}
}

\newcommand{\centeredquoteunknownsource}[1]{
	\hbadness=5000
	\vspace{-1em}
	\begin{quotation}
		\begin{flushright}
			\item\frqq\textsl{#1}\flqq\ 
		\end{flushright}
	\end{quotation}
	\vspace{-1em}
}
\usepackage[utf8]{inputenc}
\usepackage[T1]{fontenc}
\usepackage[sc,osf]{mathpazo}
\usepackage{fourier}
\usepackage{soulutf8}
\usepackage{graphicx}
\usepackage{amsmath}
\usepackage{amssymb}
\usepackage[ngerman]{babel}
\usepackage{tikz}
\usepackage{pgfplotstable}
\usepackage{ifthen}

\usepackage{qrcode}

\emergencystretch2em

\newcommand{\qr}[2][]{%
    \noindent % Verhindert Einrückung der Zeile
    \mbox{\qrcode[height=0.6in]{#2}\ifthenelse{\equal{#1}{}}{}{\textit{(#1)\quad}}} % Optionalen Text kursiv unter dem QR-Code
}

\usepackage{geometry}
\newgeometry{left=0.1cm,bottom=0.1cm,top=0.1cm,right=0.1cm}


\begin{document}

\textit{Von oben nach unten und links nach rechts}:

\begin{itemize}
	\item Das große Bild mit dem abgestürzten UFO und dem Außerirdischem unten Rechts stammt vom Youtube-Kanal von Ivan0135, von dem auch \textit{Skinny Bob} kommt. Diese Videos wurden intensiv analysiert, und es gibt ein paar Indizien dafür, dass sie Fake sind, z.\,B. wurden in einem der Skinny-Bob-Videos Schriftarten genutzt, die nicht existiert haben zur Zeit, wo es angeblich aufgenommen worden ist, und das Video wurde künstlich gealtert, und man hat die Quellen für diese künstlichen Alterungseffekte gefunden. Das Videomaterial selbst ist jedoch, trotz intensiver Bemühungen, nicht debunked. \qr[\youtube{} von ivan0135]{https://www.youtube.com/@ivan0135/} \qr[Detaillierte Analyse der Skinny-Bob-Bilder]{https://skinnybob.info/}	
	\item Der Außerirdische ganz Rechts stammt von einem \textit{4chan}-Thread. Angeblich hat der User an seinem Auto gearbeitet und dabei ein Geräusch gehört, schnell ein Bild gemacht und diesen Alien gesehen. Generell ist die UFO-Community sich einig, dass es ein Fake ist. \qr[Fandom-Wiki-Eintrag zum 4chan-Alien]{https://obscurban-legend.fandom.com/wiki/4chan_Grey_Alien}
	\item Die beiden grauen Aliens und die beiden Aliens, die nebeneinanderstehen, kommen auch von \textit{Ivan0135}. Ihr Status ist ungeklärt, sie sind bisher weder bestätigt noch debunked.
	\item Das Bild mit dem Mädchen zeigt den \textit{Solway Firth Spaceman}: Jim Templeton hat 1964 einen Ausflug mit seiner Familie gemacht und dieses Bild aufgenommen, auf dem im Hintergrund ein Astronaut zu sehen sein soll (zu erkennen am weißen Oberteil). Templeton erinnert sich nicht daran, dass da jemand gewesen wäre außer seiner Frau und seiner Tochter. Auflösung: er hat seine Frau photographiert, die er im Sucher nicht sehen konnte, weil der nur einen Ausschnitt zeigt, und sie wurde überbelichtet. \qr[Wikipedia-Artikel zum Solway Firth Spaceman]{https://en.wikipedia.org/wiki/Solway_Firth_Spaceman}
	\item Das nächste Bild zeigt den \textit{Atacama-Humanoid}, auch \textit{Ata} genannt. Er wurde 2003 von Grabräubern in der Kirche der Geisterstadt \textit{La Noria} in Chile gefunden. Er weist einige eigenartige Merkmale auf, ist nur 15\,cm groß und hat Wachstumsfugen, die impliziert, er hätte mehrere Jahre gelebt. Die UFO-Community hat diese Nachricht mit Interesse aufgenommen, es stellte sich jedoch heraus, dass es ein leider schwer behindertes Frühchen war. \qr[\wikipedia{} Atacama-Skelett]{https://de.wikipedia.org/wiki/Atacama-Skelett}
	\item Das Bild rechts zeigt \textit{Aljoshenka}, oder den \textit{Zwerg von Kyshtym}. Die Details sind vage und nicht vollständig rekonstruierbar, aber die Kurzversion ist, dass Tamara Vasilyevna Prosvirina, eine ältere, schizophrene Frau, nachts herumgewandert ist und dieses Wesen gefunden hat. Sie soll es mitgenommen und gepflegt haben, aber als sie ins Krankenhaus musste, weil sie wegen einer schizophrenen Episode auffällig geworden war, und meinte, man müsse sich um \frqq ihr Kind\flqq\ kümmern, glaubte man ihr nicht. Später fand man das Wesen verhungert, während Prosvirina in einer Anstalt lebte. Die Polizei hat die Ermittlungen eingestellt, weil sie \frqq nur für Menschen ermitteln\flqq, und der Leichnahm ist seitdem verschollen. In Kyshtym gab es 1957 einen Atomunfall, der fast so schwerwiegend war wie in Tschernobyl, aber vertuscht wurde. Höchstwahrscheinlich war \textit{Alyoshenka} ein durch Strahlen fehlgebildeter Mensch. Da aber die Leiche verschwunden ist, lässt sich das nicht mehr überprüfen. \qr[\wikipedia{} Alyoshenka]{https://en.wikipedia.org/wiki/Alyoshenka} \qr[Unheimlich-Podcast über Alyoshenka]{https://www.unheimlichpodcast.de/mystery-podcast-ep04-das-baby-alyoshenka}
	\item Das Bild links zeigt \textit{Mona Lisa}, angeblich ein Alien, der auf der geheimen \textit{Apollo 20}-Mission gefunden woren sein soll. Der französische Künstler \textit{Thierry Speth} hat sich bekannt, das Video gefaket zu haben, und hatte auch noch die Probs, die er dafür benutzt hat als Beweis. Er hat sie unter zwei Accounts hochgelden, retiredafb und moonwalker1966delta. \qr[\youtube{} retiredafb]{https://www.youtube.com/user/retiredafb} \qr[\youtube{} moonwalker1966delta]{https://www.youtube.com/@moonwalker1966delta/videos}

		\qr[Wikibooks-Eintrag über den Apollo-20-Hoax]{https://en.wikibooks.org/wiki/User:Moby-Dick4000/Apollo_20_hoax}
	\item Das nächste Alien kommt von \textit{Dr. Jonathan Reed}, der meinte, er habe diesen Alien im Wald gesehen, er habe seinen Hund angegriffen und er hat ihn mit einem Stock getötet. Es wird allgemein als Fake anerkannt.

		\qr[Fandom-Wiki zum Reed-Alien-Fall]{https://extraterrestrials.fandom.com/wiki/Reed_Case}
	\item Die nächsten beiden Bilder zeigen einen \frqq Außerirdischen\flqq, der angeblich aus Roswell stammt. Eigentlich zeigt es jedoch nur ein Modell in einem UFO-Museum (links das Bild des Modells, rechts der \frqq Alien\flqq).

		\qr[Artikel über den Roswell-Zwischenfall]{https://science.howstuffworks.com/space/aliens-ufos/history-roswell-incident.htm}
	\item Die letzte Reihe zeigt links die \textit{McPherson-Tapes}, angeblich ein Video einer echten Belagerung durch Außerirdische, ist es in Wirklichkeit ein Low-Budget-Found-Footage-Film von 1989. \qr[Fandom-Wiki-Eintrag zu den McPherson-Tapes]{https://extraterrestrials.fandom.com/wiki/McPherson_Tape}
	\item Das mittlere unterste Bild zeigt die \frqq\textit{Alien Autopsy}\flqq\ eines angeblich in Roswell gefunden Aliens. In Wahrheit ist es jedoch ein Kunstprojekt des Filmkünstlers \textit{Ray Santilli}. \qr[\wikipedia{} Santilli-Film]{https://de.wikipedia.org/wiki/Santilli-Film}
	\item Das letzte Bild zeigt einen angeblichen Alien, der in London gesichtet worden sein soll. Jedoch ist die Quellenlage äußerst schlecht, es ist alles anonym und das gesamte \frqq Video\flqq, das angeblich existieren soll, wurde nie veröffentlicht. \qr[UFO-Case-Book-Artikel über die London-Aliens]{https://www.ufocasebook.com/londonaliens.html}
\end{itemize}

\end{document}
