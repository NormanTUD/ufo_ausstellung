\documentclass{scrartcl}

% These are only some keywords for the autocompletion-feature of many editors: section, subsection, subsubsection, paragraph,
% includegraphics, width, linewidth, linespread, figure, wrapfigure, caption, label, footnote, equation, input, cite, citetitle,
% citeauthor, footfullcite, tableofcontents, printbibliography, clearpage, frq, frqq, flq, flqq, grq, grqq, glq, glqq, textit,
% texttt, mathrm, dots, pmatrix, centering, phantom, minipage, ensuremath, hfill, vfill, 

\newcommand{\centeredquote}[2]{
	\hbadness=5000
	\vspace{-1em}
	\begin{flushright}
		\item\frqq\textsl{#1}\flqq\ 
	\end{flushright}
	\nopagebreak
	\hfill ---\,\textsc{#2}\newline
	\vspace{-1em}
}

\newcommand{\centeredquoteunknownsource}[1]{
	\hbadness=5000
	\vspace{-1em}
	\begin{quotation}
		\begin{flushright}
			\item\frqq\textsl{#1}\flqq\ 
		\end{flushright}
	\end{quotation}
	\vspace{-1em}
}
\usepackage[utf8]{inputenc}
\usepackage[T1]{fontenc}
\usepackage[sc,osf]{mathpazo}
\usepackage{fourier}
\usepackage{soulutf8}
\usepackage{graphicx}
\usepackage{amsmath}
\usepackage{amssymb}
\usepackage[ngerman]{babel}
\usepackage{tikz}
\usepackage{pgfplotstable}
\usepackage{ifthen}

\usepackage{qrcode}

\emergencystretch2em

\newcommand{\qr}[2][]{%
    \noindent % Verhindert Einrückung der Zeile
    \mbox{\qrcode[height=0.6in]{#2}\ifthenelse{\equal{#1}{}}{}{\textit{(#1)\quad}}} % Optionalen Text kursiv unter dem QR-Code
}

\usepackage{geometry}
\newgeometry{left=0.1cm,bottom=0.1cm,top=0.1cm,right=0.1cm}


\begin{document}

\textit{Von oben nach unten und links nach rechts}:

\begin{itemize}
	\item Air Force Colonel Thomas J. DuBose mit den Resten des Roswell-\frq UFOs\flq, das 1947 abgestürzt sein soll. Eigentlich war es ein geheimer Ballon zur Überwachung sowietischer Atomwaffentests. Die Regierung wollte das natürlich nicht öffentlich machen und dachte, mit der Erklärung, es sei eine \frq Flying Saucer\flq\ gewesen, die gerade durch die UFO-craze bekannt wurden, würde sich das Thema schnell erledigen. Aber natürlich tat es das nicht und wurde dadurch noch mehr befeuert. Dann sind sie schnell zurückgerudert, und meinten, es sei ein Wetterballon gewesen. Dadurch wurde der Mythos geboren, die US-Regierung wisse mehr, als sie erzähle, und halte Erkenntnisse zum Thema UFO geheim.
		\qr[\wikipedia{} Thomas DuBose]{https://en.wikipedia.org/wiki/Thomas_DuBose}

		\qr[\wikipedia{} Roswell-Zwischenfall]{https://en.wikipedia.org/wiki/Roswell_incident}
		\qr[\youtube{} Internet Historian über Roswell]{https://www.youtube.com/watch?v=VT128ElBWkM}

	\item In den Jahren 1987 und 1988 soll Ed Waters in Gulf Breeze, Florida, dutzende Bilder von UFOs gemacht haben. Sie sahen sehr interessant aus, leider wurde jedoch bei ihm später ein Modell eines der UFOs auf dem Dachboden gefunden.

		\qr[\wikipedia{} Gulf-Breeze-Zwischenfall]{https://en.wikipedia.org/wiki/Gulf_Breeze_UFO_incident}
	\item Ein Bild veröffentlicht von der \frq\textit{Amalgamated Flying Saucer Clubs of America}\flq, angeblich 1963 Von Paul Villa nahe Peralta, Kalifornien (alternative Erzählungen sprechen von A. Villa Jr., der es in Albuquerque, New Mexico gesehen haben will) aufgenommen. Es wird allgemein als Fake gesehen.
		\qr[Zeitungsartikel über den Zwischenfall]{https://www.newspapers.com/article/albuquerque-journal-apolinar-a-villa-j/122308373/}
	\item Ein Bild aus der Reihe der sogenannten \textit{McMinnville-UFOs}. die 1950 von Paul und Evelyn Trent in McMinnville, Oregon, USA aufgenommen worden sein sollen. Es stellte sich heraus, dass es ein hochgeworfener Rückspiegel war.

		\qr[\wikipedia{} McMinnville-UFO-Bilder]{https://en.wikipedia.org/wiki/McMinnville_UFO_photographs}
	\item Utsurobune: Die Utsurobune-Legende erzählt von einem ungewöhnlichen Vorfall im Jahr 1803 an der Küste der japanischen Provinz Hitachi. Ein rundes, hohles Schiff trieb an Land, aus dem eine wunderschöne Frau in fremdartiger Kleidung stieg. Sie sprach eine unverständliche Sprache und trug eine geheimnisvolle Box bei sich. Dokumente der Edo-Zeit zeigen Zeichnungen des Schiffs, das in seiner Form einem modernen \frq fliegenden Untertassen\flq-Bericht ähnelt. Forscher wie Tanaka Kazuo haben diese Geschichte wegen ihrer Ähnlichkeiten zu späteren UFO-Sichtungen untersucht, obwohl die genauen Hintergründe ungeklärt bleiben. \qr[Mehr Infos über Utsurobune]{https://www.nippon.com/en/japan-topics/g00879/}
	\item Das Nürnberger Himmelsspektakel von 1561 und das Basler Himmelsspektakel von 1566 sind bemerkenswerte historische Ereignisse, die oft mit UFO-Sichtungen in Verbindung gebracht werden. In Nürnberg beobachteten die Menschen eine \frq Schlacht\flq\ am Himmel, festgehalten auf einem Flugblatt, das kugel-, kreuz- und zylinderförmige Objekte zeigte, die kämpften und zur Erde hinabstiegen. Religiöse Deutungen sahen dies als göttliches Zeichen zur Buße, während Ufologen es als frühe Sichtung von UFOs interpretieren. Meteorologen vermuten, dass natürliche Phänomene wie atmosphärische Halo-Effekte dafür verantwortlich sind. Ähnlich berichtete das Basler Flugblatt von Kugeln, die gegeneinander \frq kämpften\flq, und wurde ebenfalls als apokalyptisches Omen gedeutet. Beide Ereignisse verdeutlichen die Vermischung von natürlichen Phänomenen, religiösen Überzeugungen und spekulativen Interpretationen der damaligen Zeit.

		\qr[\wikipedia{} Basler Flugblatt]{https://de.wikipedia.org/wiki/Basler_Flugblatt_von_1566}
		\qr[\wikipedia{} Nürnberger Flugblatt]{https://de.wikipedia.org/wiki/N\%C3\%BCrnberger_Flugblatt_von_1561}
	\item \textit{Battle of Los Angeles}: Die sogenannte \frq Schlacht um Los Angeles\flq ereignete sich in der Nacht vom 24. auf den 25. Februar 1942, kurz nach dem Eintritt der USA in den Zweiten Weltkrieg. Es gab Berichte über feindliche Flugzeuge über Los Angeles, was zu einem massiven Einsatz von Flugabwehrgeschützen führte, bei dem über 1.400 Schüsse abgefeuert wurden. Später stellte sich heraus, dass es sich wahrscheinlich um einen Fehlalarm aufgrund eines Wetterballons handelte, was von Kriegsnerven und Missverständnissen verschärft wurde. Fünf Zivilisten starben durch Herzinfarkte und Unfälle während des Chaos. Obwohl der Vorfall von der Regierung als falscher Alarm erklärt wurde, führten Spekulationen und Verschwörungstheorien zu Vermutungen über eine tatsächliche Invasion oder eine Vertuschung. Der Vorfall wird heute oft als Beispiel für Panik und Überreaktion während des Krieges zitiert. UFO-Verschwörungstheoretiker interpretieren ein Foto des Ereignisses als Beweis für eine außerirdische Präsenz, obwohl das Bild stark retuschiert war. \qr[\wikipedia{} Battle of Los Angeles]{https://en.wikipedia.org/wiki/Battle_of_Los_Angeles}
	\item Bei der belgischen UFO-Welle, die 1989 begann, erlangte ein Foto eines dreieckigen UFOs mit Lichtern an den Ecken besondere Aufmerksamkeit. Es wurde im April 1990 aufgenommen und lange Zeit als Beweis für das Phänomen angesehen. Erst 2011 enthüllte Patrick Maréchal, der das Foto gemacht hatte, dass es sich um eine Fälschung handelte. Er hatte ein dreieckiges Modell aus Styropor gebaut, es schwarz bemalt und an den Ecken mit Glühbirnen versehen, bevor er es von einer Schnur hängend fotografierte. Trotz dieses Geständnisses und der Entlarvung des Bildes als Scherz hielt sich das Interesse an der UFO-Welle weiterhin. \qr[\wikipedia{} belgischen UFO-Welle]{https://en.wikipedia.org/wiki/Belgian_UFO_wave}
	\item Das \frq GIMBAL\flq-Video, das erstmals 2015 von US-Kampfpiloten aufgenommen wurde, zeigt ein scheibenförmiges, schnell fliegendes Objekt ohne sichtbare Antriebe, das sich ungewöhnlich verhält. Bereits Jahre vor der offiziellen Veröffentlichung wurde das Video auf der Website \textit{AboveTopSecret.com} geleakt, aber kaum beachtet oder ernst genommen. Erst als die New York Times 2017 darüber berichtete und das Pentagon die Echtheit der Aufnahmen bestätigte, erlangte das Video größere Aufmerksamkeit. Höchstwahrscheinlich waren es IR-Artefakte eines Flugzeug-Triebwerkes in der Entfernung.

		\qr[\youtube{} Lemmino über die Pentagon-Videos]{https://www.youtube.com/watch?v=SpeSpA3e56A}
	\item Ein angeblich 1968 in Russland abgestürztes UFO, aber eigentlich kommt das Bild aus der Doku \textit{The Secret KGB UFO Files} und wurde nur falsch benannt unter der UFO-Szene immer weiter gegeben.

		\qr[IMDB-Eintrag zur Mockumentary The Secret KGB UFO Files]{https://www.imdb.com/title/tt0224072/}
\end{itemize}

\end{document}
