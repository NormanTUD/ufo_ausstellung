\documentclass{scrartcl}

% These are only some keywords for the autocompletion-feature of many editors: section, subsection, subsubsection, paragraph,
% includegraphics, width, linewidth, linespread, figure, wrapfigure, caption, label, footnote, equation, input, cite, citetitle,
% citeauthor, footfullcite, tableofcontents, printbibliography, clearpage, frq, frqq, flq, flqq, grq, grqq, glq, glqq, textit,
% texttt, mathrm, dots, pmatrix, centering, phantom, minipage, ensuremath, hfill, vfill, 

\newcommand{\centeredquote}[2]{
	\hbadness=5000
	\vspace{-1em}
	\begin{flushright}
		\item\frqq\textsl{#1}\flqq\ 
	\end{flushright}
	\nopagebreak
	\hfill ---\,\textsc{#2}\newline
	\vspace{-1em}
}

\newcommand{\centeredquoteunknownsource}[1]{
	\hbadness=5000
	\vspace{-1em}
	\begin{quotation}
		\begin{flushright}
			\item\frqq\textsl{#1}\flqq\ 
		\end{flushright}
	\end{quotation}
	\vspace{-1em}
}
\usepackage[utf8]{inputenc}
\usepackage[T1]{fontenc}
\usepackage[sc,osf]{mathpazo}
\usepackage{fourier}
\usepackage{soulutf8}
\usepackage{graphicx}
\usepackage{amsmath}
\usepackage{amssymb}
\usepackage[ngerman]{babel}
\usepackage{tikz}
\usepackage{pgfplotstable}
\usepackage{ifthen}

\usepackage{qrcode}

\emergencystretch2em

\newcommand{\qr}[2][]{%
    \noindent % Verhindert Einrückung der Zeile
    \mbox{\qrcode[height=0.6in]{#2}\ifthenelse{\equal{#1}{}}{}{\textit{(#1)\quad}}} % Optionalen Text kursiv unter dem QR-Code
}

\usepackage{geometry}
\newgeometry{left=0.1cm,bottom=0.1cm,top=0.1cm,right=0.1cm}


\begin{document}

\textit{Von oben nach unten und links nach rechts}:

\begin{itemize}
	\item roswell
	\item In den Jahren 1987 und 1988 soll Ed Waters in Gulf Breeze, Florida, dutzende Bilder von UFOs gemacht haben. Sie sahen sehr interessant aus, leider wurde jedoch bei ihm später ein Modell eines der UFOs auf dem Dachboden gefunden.
		\qr{https://en.wikipedia.org/wiki/Gulf_Breeze_UFO_incident}
	\item Ein Bild veröffentlicht von der \frq\textit{Amalgamated Flying Saucer Clubs of America}\flq, angeblich 1963 Von Paul Villa nahe Peralta, Kalifornien (alternative Erzählungen sprechen von A. Villa Jr., der es in Albuquerque, New Mexico gesehen haben will) aufgenommen. Es wird allgemein als Fake gesehen.
		\qr{https://www.newspapers.com/article/albuquerque-journal-apolinar-a-villa-j/122308373/}
	\item Ein Bild aus der Reihe der sogenannten \textit{McMinnville-UFOs}. die 1950 von Paul und Evelyn Trent in McMinnville, Oregon, USA aufgenommen worden sein sollen. Es stellte sich heraus, dass es ein hochgeworfener Rückspiegel war.
		\qr{https://en.wikipedia.org/wiki/McMinnville_UFO_photographs}
	\item utsoburone
	\item nürnberg/altes poster
	\item battle of la
	\item triangle ufo belgien
	\item gimble/gofast
	\item Ein angeblich 1968 in Russland abgestürztes UFO, aber eigentlich kommt das Bild aus der Doku \textit{The Secret KGB UFO Files} und wurde nur falsch benannt unter der UFO-Szene immer weiter gegeben.
		\qr{https://www.imdb.com/title/tt0224072/}
\end{itemize}

\end{document}
