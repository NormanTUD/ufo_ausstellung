\documentclass{scrartcl}

% These are only some keywords for the autocompletion-feature of many editors: section, subsection, subsubsection, paragraph,
% includegraphics, width, linewidth, linespread, figure, wrapfigure, caption, label, footnote, equation, input, cite, citetitle,
% citeauthor, footfullcite, tableofcontents, printbibliography, clearpage, frq, frqq, flq, flqq, grq, grqq, glq, glqq, textit,
% texttt, mathrm, dots, pmatrix, centering, phantom, minipage, ensuremath, hfill, vfill, 

\newcommand{\centeredquote}[2]{
	\hbadness=5000
	\vspace{-1em}
	\begin{flushright}
		\item\frqq\textsl{#1}\flqq\ 
	\end{flushright}
	\nopagebreak
	\hfill ---\,\textsc{#2}\newline
	\vspace{-1em}
}

\newcommand{\centeredquoteunknownsource}[1]{
	\hbadness=5000
	\vspace{-1em}
	\begin{quotation}
		\begin{flushright}
			\item\frqq\textsl{#1}\flqq\ 
		\end{flushright}
	\end{quotation}
	\vspace{-1em}
}
\usepackage[utf8]{inputenc}
\usepackage[T1]{fontenc}
\usepackage[sc,osf]{mathpazo}
\usepackage{fourier}
\usepackage{soulutf8}
\usepackage{graphicx}
\usepackage{amsmath}
\usepackage{amssymb}
\usepackage[ngerman]{babel}
\usepackage{tikz}
\usepackage{pgfplotstable}

\usepackage{qrcode}

\emergencystretch2em

\newcommand{\qr}[1]{
	\qquad \qrcode[height=0.6in]{#1}
}

\usepackage{geometry}
\newgeometry{left=0.1cm,bottom=0.1cm,top=0.1cm,right=0.1cm}


\begin{document}

Auf diesen Bildern sind verschiedene von Raketen verursachte Phänomene zu sehen. Die Spiralen sind sogenannte Raketenspiralen. Sie entstehen auf zwei verschiedene Weisen:


\begin{itemize}
    \item Eine der Hauptursachen für das Auftreten von Raketenspiralen ist die Eigenrotation der Rakete zur Stabilisierung. Diese Rotation führt dazu, dass die Abgase spiralförmig abgegeben werden, die sich dann im Weltraum ausbreiten. Die Abgase können sich aufgrund der fehlenden Atmosphäre bis zu 250\,km groß entfalten, da im Weltraum keine Luft oder andere Hindernisse vorhanden sind, die die Ausbreitung behindern. 
    \item Eine zweite Möglichkeit sind Fehlstarts. Die Rakete \flq spiralt \frq\ sich dabei um sich selbst und stößt weiterhin Abgase ab, die ebenfalls spiralförmig sich ausbreiten.
\end{itemize}

In beiden Fällen entsteht eine sichtbare Spirale, die von der Sonne beleuchtet wird und dadurch besonders auffällig ist.

Die bekannteste Spiralformation vom 9. Dezember 2009 wurde von zahlreichen Beobachtern in den nördlichen Regionen Norwegens sowie in Teilen Schwedens gesehen. Der offizielle Grund wurde als gescheiterter Test eines russischen Bulava-Raketenmissils (ICBM) erklärt, bei dem die dritte Stufe in großer Höhe versagte. Trotz dieser Erklärung wurde weiterhin über die Herkunft und Natur des Phänomens spekuliert. Die Spirale wurde in einer Höhe zwischen 107 und 166 Meilen (172 bis 267 km) beobachtet und erreichte beim Endstadium eine Breite von etwa 629 km.

Das Phänomen wies eine unerwartete Symmetrie in den Farben und der Geometrie auf, was zu Spekulationen über alternative Erklärungen führte, darunter UFOs und HAARP-ähnliche Energievortex. Während viele Theorien diskutiert wurden, zeigte sich, dass das Spektakel eine hochentwickelte und koordinierte Erscheinung war, die durch die Umstände des misslungenen Raketentests hervorgerufen wurde. Interessanterweise fand die Expansion des spiralförmigen Phänomens mit einer Geschwindigkeit von über 12,000\,$\frac{\text{km}}{\text{h}}$) statt, was auf eine immense Dynamik hindeutet.

Eine der bekanntesten Behauptungen kam von David Wilcock, der die EISCAT-Einrichtung in Norwegen mit dem Phänomen in Verbindung brachte und darauf hinwies, dass die Symmetrie der Spirale die Theorie eines misslungenen Raketenstarts in Frage stellt. 

Die bekanntestes Raketenspirale war die \textit{2009 Norwegian Sky Spiral Anomaly}, die 2009 über Norwegen aufgetaucht war.

\qr[Wikipedia-Eintrag zur Norwegian Sky Anomaly]{https://en.wikipedia.org/wiki/2009_Norwegian_spiral_anomaly}
\qr[Analyse der Norway Spiral sky Anomaly]{https://web.archive.org/web/20210527205256/http://www.spellconsulting.com/reality/Norway_Spiral.html}

Der Begriff \textit{Space Jellyfish} (auch bekannt als \textit{Jellyfish UFO} oder \textit{Rocket Jellyfish}) beschreibt ein Phänomen, das mit Raketenstarts in Verbindung steht. Es entsteht, wenn Sonnenlicht von den hochgelegenen Abgasen des Raketenantriebs reflektiert wird, während die Rakete während der Dämmerung oder Morgendämmerung gestartet wird. Dabei befindet sich der Beobachter in der Dunkelheit, während die Abgasfahnen in großer Höhe weiterhin direktem Sonnenlicht ausgesetzt sind. Diese leuchtende Erscheinung erinnert an eine Qualle.

Die Sichtungen dieses Phänomens haben zu Panik, Ängsten vor einem nuklearen Raketenangriff und Berichten über unidentifizierte fliegende Objekte geführt. \qr[Wikipedia-Eintrag zu Space Jellyfish]{https://en.wikipedia.org/wiki/Space_jellyfish}

\end{document}
