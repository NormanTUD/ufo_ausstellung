\documentclass{scrartcl}

% These are only some keywords for the autocompletion-feature of many editors: section, subsection, subsubsection, paragraph,
% includegraphics, width, linewidth, linespread, figure, wrapfigure, caption, label, footnote, equation, input, cite, citetitle,
% citeauthor, footfullcite, tableofcontents, printbibliography, clearpage, frq, frqq, flq, flqq, grq, grqq, glq, glqq, textit,
% texttt, mathrm, dots, pmatrix, centering, phantom, minipage, ensuremath, hfill, vfill, 

\newcommand{\centeredquote}[2]{
	\hbadness=5000
	\vspace{-1em}
	\begin{flushright}
		\item\frqq\textsl{#1}\flqq\ 
	\end{flushright}
	\nopagebreak
	\hfill ---\,\textsc{#2}\newline
	\vspace{-1em}
}

\newcommand{\centeredquoteunknownsource}[1]{
	\hbadness=5000
	\vspace{-1em}
	\begin{quotation}
		\begin{flushright}
			\item\frqq\textsl{#1}\flqq\ 
		\end{flushright}
	\end{quotation}
	\vspace{-1em}
}
\usepackage[utf8]{inputenc}
\usepackage[T1]{fontenc}
\usepackage[sc,osf]{mathpazo}
\usepackage{fourier}
\usepackage{soulutf8}
\usepackage{graphicx}
\usepackage{amsmath}
\usepackage{amssymb}
\usepackage[ngerman]{babel}
\usepackage{tikz}
\usepackage{pgfplotstable}
\usepackage{ifthen}

\usepackage{qrcode}

\emergencystretch2em

\newcommand{\qr}[2][]{%
    \noindent % Verhindert Einrückung der Zeile
    \mbox{\qrcode[height=0.6in]{#2}\ifthenelse{\equal{#1}{}}{}{\textit{(#1)\quad}}} % Optionalen Text kursiv unter dem QR-Code
}

\usepackage{geometry}
\newgeometry{left=0.1cm,bottom=0.1cm,top=0.1cm,right=0.1cm}


\begin{document}

Auf diesen Bildern sind verschiedene von Raketen verursachte Phänomene zu sehen. Die Spiralen sind sogenannte Raketenspiralen. Sie entstehen auf zwei verschiedene Weisen:

\begin{itemize}
	\item Die Rakete hat eine Eigenrotation zur Stabilisierung. Dadurch werden die Abgase spiralförmig abgegeben und breiten sich aus. Manche dieser Spiralen werden bis zu 250\,km groß, denn im Weltraum ist nichts, was die Ausbreitung der Abgase verhindert.
	\item Eine zweite Möglichkeit der Entstehung sind Fehlstarks. Mitte-Rechts ist ein Bild eines fehlgeschlagenen Raketenstarts. Die Rakete \frq spiralt\flq\ sich um sich selbst, und stößt dabei weiter Abgas ab, was sich auch spiralförmig ausbreitet.
\end{itemize}

In beiden Fällen bildet sich eine sichtbare Spirale, die von der Sonne angeleuchtet wird.

Die bekanntestes Raketenspirale war die \textit{2009 Norwegian Sky Spiral Anomaly}, die 2009 über Norwegen aufgetaucht war nach einem missglücktem russischen Raketentest.
\qr{https://en.wikipedia.org/wiki/2009_Norwegian_spiral_anomaly}
\qr{https://web.archive.org/web/20210527205256/http://www.spellconsulting.com/reality/Norway_Spiral.html}

Die beiden mittleren Bilder sind sogenannte \textit{Space Jellyfish}, \frq Weltraumquallen\flq. Sie entstehen, wenn das Raketenabgas von der Sonne angestrahlt wird und sind teils hunderte Kilometer groß.
\qr{https://en.wikipedia.org/wiki/Space_jellyfish}

\end{document}
