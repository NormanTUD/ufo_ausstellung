\documentclass{scrartcl}

% These are only some keywords for the autocompletion-feature of many editors: section, subsection, subsubsection, paragraph,
% includegraphics, width, linewidth, linespread, figure, wrapfigure, caption, label, footnote, equation, input, cite, citetitle,
% citeauthor, footfullcite, tableofcontents, printbibliography, clearpage, frq, frqq, flq, flqq, grq, grqq, glq, glqq, textit,
% texttt, mathrm, dots, pmatrix, centering, phantom, minipage, ensuremath, hfill, vfill, 

\newcommand{\centeredquote}[2]{
	\hbadness=5000
	\vspace{-1em}
	\begin{flushright}
		\item\frqq\textsl{#1}\flqq\ 
	\end{flushright}
	\nopagebreak
	\hfill ---\,\textsc{#2}\newline
	\vspace{-1em}
}

\newcommand{\centeredquoteunknownsource}[1]{
	\hbadness=5000
	\vspace{-1em}
	\begin{quotation}
		\begin{flushright}
			\item\frqq\textsl{#1}\flqq\ 
		\end{flushright}
	\end{quotation}
	\vspace{-1em}
}
\usepackage[utf8]{inputenc}
\usepackage[T1]{fontenc}
\usepackage[sc,osf]{mathpazo}
\usepackage{fourier}
\usepackage{soulutf8}
\usepackage{graphicx}
\usepackage{amsmath}
\usepackage{amssymb}
\usepackage[ngerman]{babel}
\usepackage{tikz}
\usepackage{pgfplotstable}
\usepackage{ifthen}

\usepackage{qrcode}

\emergencystretch2em

\newcommand{\qr}[2][]{%
    \noindent % Verhindert Einrückung der Zeile
    \mbox{\qrcode[height=0.6in]{#2}\ifthenelse{\equal{#1}{}}{}{\textit{(#1)\quad}}} % Optionalen Text kursiv unter dem QR-Code
}

\usepackage{geometry}
\newgeometry{left=0.1cm,bottom=0.1cm,top=0.1cm,right=0.1cm}


\begin{document}

\textit{Von oben nach unten und links nach rechts}:

Die roten Blitze sind sogenannte Sprites. Sie sind höchst selten, sehr groß (bis einige hundert Kilometer groß) und sehr kurzlebig, meist nur einige Millisekunden lang. Sprites entstehen in der oberen Atmosphäre, typischerweise über Gewitterwolken, und sind das Ergebnis von elektrischen Entladungen zwischen der Atmosphäre und der Ionosphäre. Lange Zeit waren sie als Gerücht unter Piloten bekannt, die diese roten Blitze gesehen haben wollten, aber niemand wollte ihnen glauben. Erst, als die ISS sie aufgenommen hat, wurde das Phänomen ernst genommen. Sprites können in verschiedenen Formen auftreten, einschließlich diffusen und filiformen Strukturen, und können über 100 Kilometer in die Höhe reichen. \qr[Wikipedia-Eintrag zu Sprites]{https://de.wikipedia.org/wiki/Sprite_(Wetterph\%C3\%A4nomen)} \qr[Paper über Sprites]{https://agupubs.onlinelibrary.wiley.com/doi/full/10.1029/2003JA009972}

Links, das Bild mit der Wolke, zeigt einen Crown-Flash. Das ist ein Lichtstrahl, der sich bewegt, als würde er von einem Suchscheinwerfer kommen und den Himmel absuchen. In Wirklichkeit ist es Licht, das an Eiskristallen in der Atmosphäre reflektiert wird. Diese Eiskristalle richten sich anhand der elektromagnetischen Felder in den Wolken aus und verändern ihre Position, wenn sich die Felder ändern. Die Reflexion dieses Lichts erzeugt die charakteristische Erscheinung eines Crown-Flashes, der oft während heftiger Gewitter auftritt und in verschiedenen Farben wahrgenommen werden kann.

\qr[Wikipedia zu Crown-Flashes]{https://de.wikipedia.org/wiki/Crown_Flash} \qr[Video eines Crown-Flashes]{https://www.youtube.com/watch?v=zGKC1hZQSog}

Darunter ist ein sogenannter Blue Jet, ein spezieller, sehr seltener Blitztyp, der in die Höhe geht, statt zum Boden. Blue Jets entstehen ebenfalls in Verbindung mit Gewittern und treten in der oberen Atmosphäre auf, indem sie bis zu 50 Kilometer hochschießen. Ihre charakteristische blaue Farbe entsteht durch die Wechselwirkung des Lichtes mit Molekülen in der Atmosphäre, insbesondere Stickstoff. Diese Phänomene sind schwer zu beobachten, da sie nur wenige Millisekunden dauern und in der Regel mit starken Gewittern verbunden sind. \qr[Wikipedia-Eintrag über Hochatmosphärenblitze]{https://en.wikipedia.org/wiki/Upper-atmospheric_lightning}

Ganz unten ist eine sogenannte Lentikularwolke. Diese Wolken haben eine linseähnliche Form und entstehen durch die Störung der Luftströmung, wenn sie auf Berge oder andere Hindernisse trifft. Sie sind oft mit stabilen atmosphärischen Bedingungen assoziiert und können ein Indikator für kommende Wetterwechsel sein. Diese Wolken wurden vermutlich für die Form vieler UFO-Sichtungen maßgebend, da ihre ungewöhnliche Form und Erscheinung leicht mit extraterrestrischen Objekten verwechselt werden können. \qr[Wikipedia-Eintrag zu Lentikularwolken]{https://de.wikipedia.org/wiki/Lenticularis}

\end{document}
