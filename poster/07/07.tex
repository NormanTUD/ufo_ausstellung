\documentclass{scrartcl}

% These are only some keywords for the autocompletion-feature of many editors: section, subsection, subsubsection, paragraph,
% includegraphics, width, linewidth, linespread, figure, wrapfigure, caption, label, footnote, equation, input, cite, citetitle,
% citeauthor, footfullcite, tableofcontents, printbibliography, clearpage, frq, frqq, flq, flqq, grq, grqq, glq, glqq, textit,
% texttt, mathrm, dots, pmatrix, centering, phantom, minipage, ensuremath, hfill, vfill, 

\newcommand{\centeredquote}[2]{
	\hbadness=5000
	\vspace{-1em}
	\begin{flushright}
		\item\frqq\textsl{#1}\flqq\ 
	\end{flushright}
	\nopagebreak
	\hfill ---\,\textsc{#2}\newline
	\vspace{-1em}
}

\newcommand{\centeredquoteunknownsource}[1]{
	\hbadness=5000
	\vspace{-1em}
	\begin{quotation}
		\begin{flushright}
			\item\frqq\textsl{#1}\flqq\ 
		\end{flushright}
	\end{quotation}
	\vspace{-1em}
}
\usepackage[utf8]{inputenc}
\usepackage[T1]{fontenc}
\usepackage[sc,osf]{mathpazo}
\usepackage{fourier}
\usepackage{soulutf8}
\usepackage{graphicx}
\usepackage{amsmath}
\usepackage{amssymb}
\usepackage[ngerman]{babel}

\usepackage{qrcode}

\emergencystretch2em

\newcommand{\qr}[1]{
	\qquad \qrcode[height=0.4in]{#1}
}

\usepackage{geometry}
\newgeometry{left=0.1cm,bottom=0.1cm,top=0.1cm,right=0.1cm}

\begin{document}

\textit{Von oben nach unten und links nach rechts}:

Die roten Blitze sind sogenannte Sprites. Sie sind höchst selten, sehr groß (bis einige hundert Kilometer groß) und sehr kurzlebig. Lange Zeit waren sie als Gerücht unter Piloten bekannt, die diese roten Blitze gesehen haben wolen, aber niemand wollte ihnen glauben. Erst, als die ISS sie aufgenommen hat, wurde das Phänomen ernst genommen. \qr{https://de.wikipedia.org/wiki/Sprite_(Wetterph\%C3\%A4nomen)}

Links, das Bild mit der Wolke, zeigt einen Crown-Flash. Das ist ein Lichtstrahl, der sich bewegt, als würde er von einem Suchscheinwerfer kommen und den Himmel absuchen. In Wirklichkeit ist es Licht, das an Eiskristallen in der Atmosphäre reflektiert wird, die sich anhand der elektromagnetischen Felder in den Wolken ausrichten, die sich drehen, wenn sich die Felder ändern. \qr{https://de.wikipedia.org/wiki/Crown_Flash}

Darunter ein sogenannter Blue Jet, ein spezieller, sehr seltener Blitztyp, der Richtung Himmel geht statt zum Boden. \qr{https://en.wikipedia.org/wiki/Upper-atmospheric_lightning}

Ganz unten ist eine sogenannte Lentikularwolke. Das ist eine Wolke in Form einer Linse. Diese Wolken waren vermutlich für die Form der UFOs maßgebend.

\end{document}
