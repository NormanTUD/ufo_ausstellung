\documentclass{scrartcl}

% These are only some keywords for the autocompletion-feature of many editors: section, subsection, subsubsection, paragraph,
% includegraphics, width, linewidth, linespread, figure, wrapfigure, caption, label, footnote, equation, input, cite, citetitle,
% citeauthor, footfullcite, tableofcontents, printbibliography, clearpage, frq, frqq, flq, flqq, grq, grqq, glq, glqq, textit,
% texttt, mathrm, dots, pmatrix, centering, phantom, minipage, ensuremath, hfill, vfill, 

\newcommand{\centeredquote}[2]{
	\hbadness=5000
	\vspace{-1em}
	\begin{flushright}
		\item\frqq\textsl{#1}\flqq\ 
	\end{flushright}
	\nopagebreak
	\hfill ---\,\textsc{#2}\newline
	\vspace{-1em}
}

\newcommand{\centeredquoteunknownsource}[1]{
	\hbadness=5000
	\vspace{-1em}
	\begin{quotation}
		\begin{flushright}
			\item\frqq\textsl{#1}\flqq\ 
		\end{flushright}
	\end{quotation}
	\vspace{-1em}
}
\usepackage[utf8]{inputenc}
\usepackage[T1]{fontenc}
\usepackage[sc,osf]{mathpazo}
\usepackage{fourier}
\usepackage{soulutf8}
\usepackage{graphicx}
\usepackage{amsmath}
\usepackage{amssymb}
\usepackage[ngerman]{babel}
\usepackage{tikz}
\usepackage{pgfplotstable}
\usepackage{ifthen}

\usepackage{qrcode}

\emergencystretch2em

\newcommand{\qr}[2][]{%
    \noindent % Verhindert Einrückung der Zeile
    \mbox{\qrcode[height=0.6in]{#2}\ifthenelse{\equal{#1}{}}{}{\textit{(#1)\quad}}} % Optionalen Text kursiv unter dem QR-Code
}

\usepackage{geometry}
\newgeometry{left=0.1cm,bottom=0.1cm,top=0.1cm,right=0.1cm}


\begin{document}

\textit{Von oben nach unten und links nach rechts}:

\begin{itemize}
	\item Das Avrocar, ein militärisches Testflugzeug in Form eines UFOs. \qr{https://de.wikipedia.org/wiki/Avro_Canada_VZ-9AV}
	\item Das Triangle-UFO, das von einem militärischen Schiff geleakt wurde. Eigentlich waren es die Out-of-Focus seienden Warnlichter eines Flugzeuges.
	\item Rechts daneben Paul Bennewitz: ein amerikanischer Pilot, der in der Nähe eines Stützpunktes, auf dem geheime Tests durchgeführt worden sind, von der US-Armee manipuliert wurde, damit ihm niemand glaubt. Ihm wurden falsche Probs in die Wüste gelegt und er wurde darüber geflogen, und \frq eingeweiht\flq\ in die Geheimnisse des US-Militärs. Sein Computer wurde manipuliert, so dass er Nachrichten \frq aus der Basis\flq\ empfangen hat, die immer abstruser wurden. So berichteten sie davon, dass die Außerirdischen in Dulce in einer Untergrundbasis gegen Menschen gekämpft und gewonnen haben. Bennewitz wurde in eine psychiatrische Klinik eingewiesen und starb dort vereinsamt und verarmt.
	\item Darunter das Titelblatt von \textit{Project Grudge}, einem geheimen Forschungsprojekt der US-Armee, die versucht haben, UFO-Sichtungen zu erforschen.
	\item Rechts daneben General John A. Samford, der nach Roswell eine Pressekonferenz über den Absturz gegeben hat.
	\item Darunter ein Flare, eine militärische Leuchtrakete, die oft für UFO-Sichtungen sorgt.
\end{itemize}

\end{document}
