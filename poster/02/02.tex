\documentclass{scrartcl}

% These are only some keywords for the autocompletion-feature of many editors: section, subsection, subsubsection, paragraph,
% includegraphics, width, linewidth, linespread, figure, wrapfigure, caption, label, footnote, equation, input, cite, citetitle,
% citeauthor, footfullcite, tableofcontents, printbibliography, clearpage, frq, frqq, flq, flqq, grq, grqq, glq, glqq, textit,
% texttt, mathrm, dots, pmatrix, centering, phantom, minipage, ensuremath, hfill, vfill, 

\newcommand{\centeredquote}[2]{
	\hbadness=5000
	\vspace{-1em}
	\begin{flushright}
		\item\frqq\textsl{#1}\flqq\ 
	\end{flushright}
	\nopagebreak
	\hfill ---\,\textsc{#2}\newline
	\vspace{-1em}
}

\newcommand{\centeredquoteunknownsource}[1]{
	\hbadness=5000
	\vspace{-1em}
	\begin{quotation}
		\begin{flushright}
			\item\frqq\textsl{#1}\flqq\ 
		\end{flushright}
	\end{quotation}
	\vspace{-1em}
}
\usepackage[utf8]{inputenc}
\usepackage[T1]{fontenc}
\usepackage[sc,osf]{mathpazo}
\usepackage{fourier}
\usepackage{soulutf8}
\usepackage{graphicx}
\usepackage{amsmath}
\usepackage{amssymb}
\usepackage[ngerman]{babel}
\usepackage{tikz}
\usepackage{pgfplotstable}
\usepackage{ifthen}

\usepackage{qrcode}

\emergencystretch2em

\newcommand{\qr}[2][]{%
    \noindent % Verhindert Einrückung der Zeile
    \mbox{\qrcode[height=0.6in]{#2}\ifthenelse{\equal{#1}{}}{}{\textit{(#1)\quad}}} % Optionalen Text kursiv unter dem QR-Code
}

\usepackage{geometry}
\newgeometry{left=0.1cm,bottom=0.1cm,top=0.1cm,right=0.1cm}


\begin{document}

Im März 1971 wurden angeblich von dem US-U-Boot \frq USS Trepang\flq\ Aufnahmen durch das Periskop gemacht, die kürzlich für Aufregung unter UFO-Forschern und Skeptikern sorgten. Die Bilder zeigen verschiedene Flugobjekte über der Wasseroberfläche, die entweder aus dem Wasser aufsteigen oder nach Beschuss ins Wasser stürzen. Die UFO-Forscher Alex Mistretta, John Greenewald Jr. und Steve Murillo haben die Hintergründe der Bilder untersucht und glauben, eine Erklärung gefunden zu haben.

Mistretta hörte von den Aufnahmen erstmals durch eine vertrauenswürdige Informationsquelle in Europa, die bestätigte, dass es sich um eine Fotoserie handelt, die zuvor im französischen Magazin \frq Top Secret\flq\ veröffentlicht wurde. Laut dieser anonymen Quelle wurden die Bilder von Bord der \frq USS Trepang\flq\ im Arktischen Meer zwischen Island und der norwegischen Insel Jan Mayen aufgenommen, als Admiral Dean Reynolds Sackett das Kommando hatte. Der Offizier John Klika soll die Objekte zufällig mit dem Periskop entdeckt haben.

Die von Mistretta erhaltenen Bilder wiesen jedoch keine Aufschriften auf, während das Magazin \frq Top Secret\flq\ behauptete, dass bestimmte Vermerke zu finden seien. Mistretta erhielt am 13. Juli 2015 Originalabzüge der Fotoserie ohne diese Aufschriften, die nicht Scans oder Kopien waren. Recherchen bestätigten, dass sowohl Admiral Sackett als auch Klika zur fraglichen Zeit auf der Trepang waren, doch beide berichteten, nichts Ungewöhnliches gesehen zu haben. Klika äußerte, dass er nicht wisse, um was es sich bei den Objekten auf den Bildern handeln könnte.

Greenewald stellte fest, dass die Bilder möglicherweise Zielballons der Marine zeigen, die während Waffentests verwendet wurden. Offizielle Marineunterlagen belegen, dass zwischen dem 22. Februar und dem 22. März 1971 U-Boote unter der nördlichen Eiskappe Tests zur Datengewinnung für Waffensysteme durchführten. Historische Aufnahmen von Ballon-Trägern aus den Jahren 1910-1915 zeigen ähnliche Objekte, was die Theorie unterstützt, dass es sich um militärische Testziele handelt.

Greenewald vermutet, dass die Aufnahmen wahrscheinlich Teil von Waffentests waren und die UFO-artigen Objekte als Testziele dienten, ähnlich denen, die in der frühen Luftfahrt eingesetzt wurden. Diese Ereignisse verdeutlichen, wie militärische Aktivitäten leicht zu Spekulationen über UFOs führen können, und werfen Fragen zur Art und Weise auf, wie Informationen über solche Vorfälle vermittelt und interpretiert werden. \qr{https://www.theblackvault.com/casefiles/arctic-ufo-photographs-uss-trepang-ssn-674-march-1971/} \qr{https://www.grenzwissenschaft-aktuell.de/erklaerung-fuer-arktis-ufo-fotos-der-uss-trepang20150727/}

\end{document}
