\documentclass{scrartcl}

% These are only some keywords for the autocompletion-feature of many editors: section, subsection, subsubsection, paragraph,
% includegraphics, width, linewidth, linespread, figure, wrapfigure, caption, label, footnote, equation, input, cite, citetitle,
% citeauthor, footfullcite, tableofcontents, printbibliography, clearpage, frq, frqq, flq, flqq, grq, grqq, glq, glqq, textit,
% texttt, mathrm, dots, pmatrix, centering, phantom, minipage, ensuremath, hfill, vfill, 

\newcommand{\centeredquote}[2]{
	\hbadness=5000
	\vspace{-1em}
	\begin{flushright}
		\item\frqq\textsl{#1}\flqq\ 
	\end{flushright}
	\nopagebreak
	\hfill ---\,\textsc{#2}\newline
	\vspace{-1em}
}

\newcommand{\centeredquoteunknownsource}[1]{
	\hbadness=5000
	\vspace{-1em}
	\begin{quotation}
		\begin{flushright}
			\item\frqq\textsl{#1}\flqq\ 
		\end{flushright}
	\end{quotation}
	\vspace{-1em}
}
\usepackage[utf8]{inputenc}
\usepackage[T1]{fontenc}
\usepackage[sc,osf]{mathpazo}
\usepackage{fourier}
\usepackage{soulutf8}
\usepackage{graphicx}
\usepackage{amsmath}
\usepackage{amssymb}
\usepackage[ngerman]{babel}
\usepackage{tikz}
\usepackage{pgfplotstable}
\usepackage{ifthen}

\usepackage{qrcode}

\emergencystretch2em

\newcommand{\qr}[2][]{%
    \noindent % Verhindert Einrückung der Zeile
    \mbox{\qrcode[height=0.6in]{#2}\ifthenelse{\equal{#1}{}}{}{\textit{(#1)\quad}}} % Optionalen Text kursiv unter dem QR-Code
}

\usepackage{geometry}
\newgeometry{left=0.1cm,bottom=0.1cm,top=0.1cm,right=0.1cm}


\begin{document}

Dieses Bild zeigt \textit{Skytracker} und \textit{LIDAR}s.

Skytracker, auch bekannt als Himmelsstrahler oder Himmelsscheinwerfer, sind am bekanntesten vor Discos und auf großen Veranstaltungen. Sie strahlen sehr helle Strahlen an den Himmel, die, wenn sie sich an den Wolken reflektieren, eigenartig aussehen können und oft für UFO-Sichtungen sorgen. \qr{https://de.wikipedia.org/wiki/Himmelsstrahler}

LIDAR steht für \textit{\textbf{Li}ght \textbf{D}etection and \textbf{R}anging} und ist eine Methode, um Distanzen mithilfe eines Lasers festzustellen. Sie werden z.\,B. von Sternwarten genutzt, um die Entfernung zum Mond zu bestimmen. \qr{https://en.wikipedia.org/wiki/Lidar} \qr{https://de.wikipedia.org/wiki/Lunar_Laser_Ranging}.

\end{document}
