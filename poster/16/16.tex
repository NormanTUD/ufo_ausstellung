\documentclass{scrartcl}

% These are only some keywords for the autocompletion-feature of many editors: section, subsection, subsubsection, paragraph,
% includegraphics, width, linewidth, linespread, figure, wrapfigure, caption, label, footnote, equation, input, cite, citetitle,
% citeauthor, footfullcite, tableofcontents, printbibliography, clearpage, frq, frqq, flq, flqq, grq, grqq, glq, glqq, textit,
% texttt, mathrm, dots, pmatrix, centering, phantom, minipage, ensuremath, hfill, vfill, 

\newcommand{\centeredquote}[2]{
	\hbadness=5000
	\vspace{-1em}
	\begin{flushright}
		\item\frqq\textsl{#1}\flqq\ 
	\end{flushright}
	\nopagebreak
	\hfill ---\,\textsc{#2}\newline
	\vspace{-1em}
}

\newcommand{\centeredquoteunknownsource}[1]{
	\hbadness=5000
	\vspace{-1em}
	\begin{quotation}
		\begin{flushright}
			\item\frqq\textsl{#1}\flqq\ 
		\end{flushright}
	\end{quotation}
	\vspace{-1em}
}
\usepackage[utf8]{inputenc}
\usepackage[T1]{fontenc}
\usepackage[sc,osf]{mathpazo}
\usepackage{fourier}
\usepackage{soulutf8}
\usepackage{graphicx}
\usepackage{amsmath}
\usepackage{amssymb}
\usepackage[ngerman]{babel}
\usepackage{tikz}
\usepackage{pgfplotstable}
\usepackage{ifthen}

\usepackage{qrcode}

\emergencystretch2em

\newcommand{\qr}[2][]{%
    \noindent % Verhindert Einrückung der Zeile
    \mbox{\qrcode[height=0.6in]{#2}\ifthenelse{\equal{#1}{}}{}{\textit{(#1)\quad}}} % Optionalen Text kursiv unter dem QR-Code
}

\usepackage{geometry}
\newgeometry{left=0.1cm,bottom=0.1cm,top=0.1cm,right=0.1cm}


\begin{document}

Dieses Bild zeigt \textit{Skytracker} und \textit{LIDAR}s.

Skytracker, oder auch Himmelsstrahler genannt, sind diese beeindruckenden Lichtstrahlen, die man oft vor Discos und bei großen Veranstaltungen sieht. Sie werfen extrem helle, fokussierte Strahlen in den Himmel, die beim Auftreffen auf Wolken eine spektakuläre Show bieten. Wenn die Strahlen auf die Wolken treffen, können sie bizarre Muster und Formen erzeugen, die manchmal einfach atemberaubend sind.

Diese Lichtspiele sind nicht nur ein Fest für die Augen, sondern haben auch schon für einige UFO-Sichtungen gesorgt. Manchmal wird der Himmel zu einem wabernden Meer aus Licht, und das kann leicht den Eindruck erwecken, dass da oben etwas Unheimliches vor sich geht. Tatsächlich sind viele \frq UFO-Sichtungen\flq\ in städtischen Gebieten oft auf diese Himmelsstrahler zurückzuführen. Leute schauen nach oben und denken, sie hätten etwas Übernatürliches gesehen, dabei ist es nur ein sehr kreativer Lichtstrahler! \qr[Wikipedia-Eintrag zu Himmelsstrahlern]{https://de.wikipedia.org/wiki/Himmelsstrahler}.

LIDAR, das steht für \textit{\textbf{Li}ght \textbf{D}etection and \textbf{R}anging}, ist eine faszinierende Technologie, die Laserstrahlen nutzt, um Entfernungen zu messen. Wie funktioniert das? Ganz einfach: Ein Laser sendet einen Strahl aus, und wenn dieser auf ein Objekt trifft, reflektiert er zurück. Durch die Messung der Zeit, die der Laser benötigt, um zurückzukommen, kann man die Entfernung präzise bestimmen.

In der Astronomie wird LIDAR zum Beispiel verwendet, um die Entfernung zum Mond zu messen. Man sendet einen Laserstrahl in Richtung Mond, und die Zeit, die der Strahl braucht, um zurückzukommen, verrät uns, wie weit der Mond entfernt ist. Diese Technologie hat auch ihre eigene Verbindung zur UFO-Forschung: Einige LIDAR-Systeme können potenzielle Objekte im Himmel identifizieren, was bedeutet, dass sie manchmal auch als „Radar“ für UFO-Jäger genutzt werden!

Zusätzlich zur präzisen Entfernungsmessung kann LIDAR auch genutzt werden, um die Form und Struktur von Objekten zu analysieren. Es kann also helfen, die Umgebung genau zu kartieren und sogar etwas wie die Höhenlage von Wolken zu bestimmen. So wird die Technologie nicht nur für wissenschaftliche Zwecke verwendet, sondern bietet auch spannende Möglichkeiten, um unerklärliche Phänomene am Himmel zu untersuchen. \qr[Wikipedia-Eintrag zu LIDAR]{https://en.wikipedia.org/wiki/Lidar} \qr[Wikipedia-Eintrag zum Lunar Laser Ranging]{https://de.wikipedia.org/wiki/Lunar_Laser_Ranging}

\end{document}
