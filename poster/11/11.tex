\documentclass{scrartcl}

% These are only some keywords for the autocompletion-feature of many editors: section, subsection, subsubsection, paragraph,
% includegraphics, width, linewidth, linespread, figure, wrapfigure, caption, label, footnote, equation, input, cite, citetitle,
% citeauthor, footfullcite, tableofcontents, printbibliography, clearpage, frq, frqq, flq, flqq, grq, grqq, glq, glqq, textit,
% texttt, mathrm, dots, pmatrix, centering, phantom, minipage, ensuremath, hfill, vfill, 

\newcommand{\centeredquote}[2]{
	\hbadness=5000
	\vspace{-1em}
	\begin{flushright}
		\item\frqq\textsl{#1}\flqq\ 
	\end{flushright}
	\nopagebreak
	\hfill ---\,\textsc{#2}\newline
	\vspace{-1em}
}

\newcommand{\centeredquoteunknownsource}[1]{
	\hbadness=5000
	\vspace{-1em}
	\begin{quotation}
		\begin{flushright}
			\item\frqq\textsl{#1}\flqq\ 
		\end{flushright}
	\end{quotation}
	\vspace{-1em}
}
\usepackage[utf8]{inputenc}
\usepackage[T1]{fontenc}
\usepackage[sc,osf]{mathpazo}
\usepackage{fourier}
\usepackage{soulutf8}
\usepackage{graphicx}
\usepackage{amsmath}
\usepackage{amssymb}
\usepackage[ngerman]{babel}
\usepackage{tikz}
\usepackage{pgfplotstable}

\usepackage{qrcode}

\emergencystretch2em

\newcommand{\qr}[1]{
	\qquad \qrcode[height=0.6in]{#1}
}

\usepackage{geometry}
\newgeometry{left=0.1cm,bottom=0.1cm,top=0.1cm,right=0.1cm}


\begin{document}

\textit{Von oben nach unten und links nach rechts}:

\begin{itemize}
	\item Linda Moulton Howe: Eine amerikanische Journalistin, die dafür bekannt ist, sich mit dem UFO-Phänomen, und darunter speziell Tierverstümmelungen, zu befassen. Sie ist überzeugt, die Tierverstümmelungen stammen sicher von Außerirdischen. \qr[\wikipedia{} Linda Moulton Howe]{https://en.wikipedia.org/wiki/Linda_Moulton_Howe}
	\item Charles Hickson und Calvin Parker berichteten, dass sie am 11. Oktober 1973 beim Angeln an der Pascagoula River in Mississippi von einem UFO entführt wurden. Laut ihren Aussagen hörten sie ein seltsames Geräusch und sahen ein ovales Objekt mit blinkenden Lichtern. Während sie \frqq bewusst, aber gelähmt\flqq\ waren, wurden sie von drei ungewöhnlichen Wesen mit roboterhaften Merkmalen an Bord des UFOs gebracht. Dort behaupteten sie, einer Untersuchung unterzogen worden zu sein.

Nach dem Vorfall gaben sie mehrere Interviews, und Hickson veröffentlichte ein Buch über die Erfahrung. Das Ereignis zog viel Aufmerksamkeit auf sich und führte zu Spekulationen über UFOs und mögliche Alien-Kontakte. Skeptiker hinterfragten ihre Geschichten, wiesen auf Widersprüche hin und stellten die Authentizität ihrer Aussagen in Frage. Trotz der Kontroversen bleibt der Pascagoula-Fall ein bedeutendes Ereignis in der UFO-Geschichte.

		\qr[\wikipedia{} Pascagoula-Entführung]{https://en.wikipedia.org/wiki/Pascagoula_Abduction}
		\qr[\youtube{} GEP-Video über den Pascagoula-Fall]{https://www.youtube.com/watch?v=gqJEE5u5QRM}
	\item Art Bell: Ein amerikanischer Radio-Host, vorallem bekannt für die Sendung \textit{Coast-to-Coast-AM}, in der Anrufer mysteriöse Geschichten erzählten. Darunter waren unter Anderem Mels Hole, aber auch viele UFO-relatierte Geschichten wie die Californian Drones oder ein angeblicher Anrufer, der in der Area 51 gearbeitet haben will.
		\qr[\wikipedia{} Art Bell]{https://en.wikipedia.org/wiki/Art_Bell}
	\item Vogel: Ein verschwommenes Bild eines Vogels, der als UFO fehlklassifiziert wurde.
	\item London Alien: Angebliche Aliens, die in London aufgenommen worden sein sollen.
		\qr[UFO-Case-Book Eintrag]{https://www.ufocasebook.com/londonaliens.html}
\end{itemize}

\end{document}
