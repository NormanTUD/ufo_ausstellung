\documentclass{scrartcl}

% These are only some keywords for the autocompletion-feature of many editors: section, subsection, subsubsection, paragraph,
% includegraphics, width, linewidth, linespread, figure, wrapfigure, caption, label, footnote, equation, input, cite, citetitle,
% citeauthor, footfullcite, tableofcontents, printbibliography, clearpage, frq, frqq, flq, flqq, grq, grqq, glq, glqq, textit,
% texttt, mathrm, dots, pmatrix, centering, phantom, minipage, ensuremath, hfill, vfill, 

\newcommand{\centeredquote}[2]{
	\hbadness=5000
	\vspace{-1em}
	\begin{flushright}
		\item\frqq\textsl{#1}\flqq\ 
	\end{flushright}
	\nopagebreak
	\hfill ---\,\textsc{#2}\newline
	\vspace{-1em}
}

\newcommand{\centeredquoteunknownsource}[1]{
	\hbadness=5000
	\vspace{-1em}
	\begin{quotation}
		\begin{flushright}
			\item\frqq\textsl{#1}\flqq\ 
		\end{flushright}
	\end{quotation}
	\vspace{-1em}
}
\usepackage[utf8]{inputenc}
\usepackage[T1]{fontenc}
\usepackage[sc,osf]{mathpazo}
\usepackage{fourier}
\usepackage{soulutf8}
\usepackage{graphicx}
\usepackage{amsmath}
\usepackage{amssymb}
\usepackage[ngerman]{babel}
\usepackage{tikz}
\usepackage{pgfplotstable}
\usepackage{ifthen}

\usepackage{qrcode}

\emergencystretch2em

\newcommand{\qr}[2][]{%
    \noindent % Verhindert Einrückung der Zeile
    \mbox{\qrcode[height=0.6in]{#2}\ifthenelse{\equal{#1}{}}{}{\textit{(#1)\quad}}} % Optionalen Text kursiv unter dem QR-Code
}

\usepackage{geometry}
\newgeometry{left=0.1cm,bottom=0.1cm,top=0.1cm,right=0.1cm}


\begin{document}

\textit{Von oben nach unten und links nach rechts}:

\begin{itemize}
	\item Linda Moulton Howe: Eine amerikanische Journalistin, die dafür bekannt ist, sich mit dem UFO-Phänomen, und darunter speziell Tierverstümmelungen, zu befassen. Sie ist überzeugt, die Tierverstümmelungen stammen sicher von Außerirdischen. \qr[\wikipedia{} Linda Moulton Howe]{https://en.wikipedia.org/wiki/Linda_Moulton_Howe}
	\item Charles Hickson und Calvin Parker: sie behaupteten, 1973 nahe der Stadt Pascagoula von Außerirdischen entführt worden zu sein.
		\qr[\wikipedia{} Pascagoula-Entführung]{https://en.wikipedia.org/wiki/Pascagoula_Abduction}
		\qr[\youtube{} GEP-Video über den Pascagoula-Fall]{https://www.youtube.com/watch?v=gqJEE5u5QRM}
	\item Art Bell: Ein amerikanischer Radio-Host, vorallem bekannt für die Sendung \textit{Coast-to-Coast-AM}, in der Anrufer mysteriöse Geschichten erzählten. Darunter waren unter Anderem Mels Hole, aber auch viele UFO-relatierte Geschichten wie die Californian Drones oder ein angeblicher Anrufer, der in der Area 51 gearbeitet haben will.
		\qr[\wikipedia{} Art Bell]{https://en.wikipedia.org/wiki/Art_Bell}
	\item Vogel: Ein verschwommenes Bild eines Vogels, der als UFO fehlklassifiziert wurde.
	\item London Alien: Angebliche Aliens, die in London aufgenommen worden sein sollen.
\end{itemize}

\end{document}
